\section*{Câu 4}

\subsection*{Thông số mạch điện:}

Dựa vào Hình 4 và quy tắc $@ = 7$ bạn đã cung cấp:

\begin{itemize}
    \item Op-Amp lý tưởng.
    \item $R_1 = 20 \text{ k}\Omega$.
    \item $R_2 = 56 \text{ k}\Omega$.
    \item $R_3 = 10 \text{ k}\Omega$.
    \item $R_4 = 3@ \text{ k}\Omega = 37 \text{ k}\Omega$.
\end{itemize}

\subsection*{a) Xác định điện áp ngõ ra $v_O$}

Trường hợp: $v_{I1} = 1 \text{ V}$ và $v_{I2} = -1 \text{ V}$.

Đây là mạch khuếch đại sai biệt (differential amplifier). Điện áp ngõ ra $v_O$ được tính theo công thức tổng quát:

$$v_O = \left( \frac{R_4}{R_3 + R_4} \right) \left( 1 + \frac{R_2}{R_1} \right) v_{I2} - \left( \frac{R_2}{R_1} \right) v_{I1}$$

\textbf{Thay các giá trị điện trở vào công thức:}

Tỉ số $\frac{R_2}{R_1} = \frac{56}{20} = 2.8$.

Hệ số phân áp tại ngõ vào không đảo: $\frac{R_4}{R_3 + R_4} = \frac{37}{10 + 37} = \frac{37}{47}$.

\textbf{Thay các giá trị điện áp ngõ vào $v_{I1}, v_{I2}$:}

$$v_O = \left( \frac{37}{47} \right) (1 + 2.8) (-1) - (2.8) (1)$$

$$v_O = \left( \frac{37}{47} \right) (3.8) (-1) - 2.8$$

$$v_O \approx (0.7872) (3.8) (-1) - 2.8$$

$$v_O \approx -2.991 - 2.8 = -5.791 \text{ V}$$

Kết quả: $v_O \approx \mathbf{-5.79 \text{ V}}$.

\subsection*{b) Xác định tỉ số độ lợi cách chung $A_{cm}$}

Trường hợp: $v_{I1} = v_{I2} = v_I$ (tín hiệu cách chung).

Tỉ số độ lợi cách chung là $A_{cm} = \frac{v_O}{v_I}$.

Thay $v_{I1} = v_{I2} = v_I$ vào công thức tổng quát của $v_O$:

$$v_O = \left[ \left( \frac{R_4}{R_3 + R_4} \right) \left( 1 + \frac{R_2}{R_1} \right) - \frac{R_2}{R_1} \right] v_I$$

Từ đó, ta có biểu thức cho $A_{cm}$:

$$A_{cm} = \left( \frac{R_4}{R_3 + R_4} \right) \left( 1 + \frac{R_2}{R_1} \right) - \frac{R_2}{R_1}$$

\textbf{Thay số:}

$$A_{cm} = \left( \frac{37}{47} \right) (3.8) - 2.8$$

$$A_{cm} = \frac{140.6}{47} - 2.8 \approx 2.9915 - 2.8 = 0.1915$$

Để chính xác hơn, ta dùng phân số:

$$A_{cm} = \frac{37}{47} \cdot \frac{19}{5} - \frac{14}{5} = \frac{703}{235} - \frac{658}{235} = \frac{45}{235} = \frac{9}{47}$$

$$A_{cm} = \frac{9}{47} \approx 0.191$$

Kết quả: $A_{cm} = \frac{9}{47} \approx \mathbf{0.191}$.

\subsection*{c) Tính tỉ số loại bỏ chế độ cùng pha CMRR}

Công thức: $\text{CMRR} = |A_d / A_{cm}|$.

Trước hết, ta cần tính độ lợi sai biệt $A_d$. Từ định nghĩa $v_O = A_d(v_{I2} - v_{I1}) + A_{cm}(v_{I1} + v_{I2})/2$, ta có thể suy ra:

$$A_d = \frac{1}{2} \left[ \left( \frac{R_4}{R_3 + R_4} \right) \left( 1 + \frac{R_2}{R_1} \right) + \frac{R_2}{R_1} \right]$$

\textbf{Thay số:}

$$A_d = \frac{1}{2} \left[ \left( \frac{37}{47} \right) (3.8) + 2.8 \right]$$

$$A_d = \frac{1}{2} [ 2.9915 + 2.8 ] = \frac{1}{2} [ 5.7915 ] = 2.89575$$

Sử dụng phân số để tính chính xác:

$$A_d = \frac{1}{2} \left( \frac{703}{235} + \frac{658}{235} \right) = \frac{1}{2} \left( \frac{1361}{235} \right) = \frac{1361}{470}$$

Bây giờ tính CMRR:

$$\text{CMRR} = \frac{|A_d|}{|A_{cm}|} = \frac{1361/470}{9/47} = \frac{1361}{470} \cdot \frac{47}{9} = \frac{1361}{90}$$

$$\text{CMRR} \approx 15.12$$

Kết quả: $\text{CMRR} = \frac{1361}{90} \approx \mathbf{15.1}$.

\subsection*{d) Sửa đổi mạch để có $v_O = A_d(v_{I2} - v_{I1})$}

Để ngõ ra chỉ phụ thuộc vào hiệu điện thế hai ngõ vào ($A_{cm} = 0$), mạch cầu điện trở phải cân bằng. Điều kiện cân bằng là:

$$\frac{R_1}{R_2} = \frac{R_3}{R_4} \quad \text{hay} \quad R_1 R_4 = R_2 R_3$$

Kiểm tra với giá trị hiện tại: $20 \cdot 37 = 740$ và $56 \cdot 10 = 560$. Mạch chưa cân bằng.

\textbf{Giải pháp sử dụng thêm 1 điện trở $R_5$:}

Ta có thể mắc thêm $R_5$ nối tiếp với $R_3$ để tăng giá trị nhánh này, sao cho đẳng thức được thỏa mãn.

Gọi $R_{3\_mới} = R_3 + R_5$. Điều kiện cân bằng mới là:

$$R_1 R_4 = R_2 (R_3 + R_5)$$

\textbf{Tính giá trị $R_5$:}

$$20 \text{ k}\Omega \cdot 37 \text{ k}\Omega = 56 \text{ k}\Omega \cdot (10 \text{ k}\Omega + R_5)$$

$$740 = 56 (10 + R_5)$$

$$10 + R_5 = \frac{740}{56} = \frac{185}{14}$$

$$R_5 = \frac{185}{14} - 10 = \frac{185 - 140}{14} = \frac{45}{14} \text{ k}\Omega$$

$$R_5 \approx 3.214 \text{ k}\Omega$$

\textbf{Sơ đồ mạch mới:}

Mắc thêm điện trở $R_5 = 3.21 \text{ k}\Omega$ nối tiếp với điện trở $R_3$ tại ngõ vào $v_{I2}$.

\textbf{Độ lợi áp $A_d$ trong trường hợp này:}

Khi mạch cân bằng ($\frac{R_1}{R_2} = \frac{R_{3\_mới}}{R_4}$), độ lợi sai biệt được tính đơn giản là:

$$A_d = \frac{R_2}{R_1} = \frac{56}{20} = 2.8$$

\textbf{Kết quả:}
\begin{itemize}
    \item Mạch: Mắc nối tiếp $R_5$ với $R_3$.
    \item Giá trị: $R_5 = \frac{45}{14} \text{ k}\Omega \approx \mathbf{3.21 \text{ k}\Omega}$.
    \item Độ lợi $A_d$ mới: $A_d = \mathbf{2.8}$.
\end{itemize}

