\section*{Câu 2}

\subsection*{Thông số đã cho:}
\begin{itemize}
    \item FET loại NMOS.
    \item $V_{DD} = +15$ V.
    \item $V_t = 1.5$ V.
    \item $k_n = 0.25$ mA/V$^2$.
    \item $V_A = 5@$ V. Với $@ = 7$, ta có $V_A = 57$ V.
    \item $R_G = 10$ M$\Omega$.
    \item $R_D = 10$ k$\Omega$.
    \item $R_L = 10$ k$\Omega$.
\end{itemize}

\subsection*{a) Tính giá trị các dòng của FET ở chế độ DC}

Ở chế độ DC, các tụ điện xem như hở mạch. Cực Gate (G) của MOSFET có trở kháng vô cùng lớn nên dòng điện $I_G = 0$.

\textbf{Xác định mối quan hệ giữa các điện áp:}

Do $I_G = 0$, không có sụt áp trên điện trở $R_G$. Vì vậy, điện áp tại cực Gate bằng điện áp tại cực Drain: $V_G = V_D$.

Cực Source (S) nối đất nên $V_S = 0$ V.

Suy ra: $V_{GS} = V_G - V_S = V_G$ và $V_{DS} = V_D - V_S = V_D$.

Do đó, ta có $V_{GS} = V_{DS}$.

\textbf{Kiểm tra vùng hoạt động:}

Điều kiện để MOSFET hoạt động trong vùng bão hòa (saturation) là $V_{GS} > V_t$ và $V_{DS} \ge V_{GS} - V_t$.

Vì $V_{DS} = V_{GS}$ và $V_t = 1.5 > 0$, điều kiện thứ hai ($V_{GS} \ge V_{GS} - 1.5$) luôn được thỏa mãn. Ta giả sử transistor dẫn ($V_{GS} > V_t$) và hoạt động trong vùng bão hòa.

\textbf{Thiết lập và giải phương trình:}

Phương trình dòng điện trong vùng bão hòa (bỏ qua hiệu ứng điều biến kênh dẫn ở chế độ DC):

$$I_D = k_n(V_{GS} - V_t)^2 = k_n(V_D - V_t)^2$$

Phương trình đường tải ngõ ra (vòng $V_{DD} - R_D - D - S$):

$$V_{DD} = I_D R_D + V_{DS} \Rightarrow 15 = I_D(10) + V_D \Rightarrow V_D = 15 - 10I_D$$

(với $I_D$ tính bằng mA, $R_D$ bằng k$\Omega$).

Thay $V_D$ vào phương trình dòng điện:

$$I_D = 0.25 ( (15 - 10I_D) - 1.5 )^2 = 0.25 (13.5 - 10I_D)^2$$

$$4I_D = (13.5 - 10I_D)^2$$

Lấy căn bậc hai hai vế: $2\sqrt{I_D} = \pm(13.5 - 10I_D)$.

Xét trường hợp $2\sqrt{I_D} = 13.5 - 10I_D$:

Đặt $x = \sqrt{I_D}$ ($x>0$), ta có phương trình: $10x^2 + 2x - 13.5 = 0$.

Giải phương trình bậc 2: 

$$x = \frac{-2 + \sqrt{4 - 4(10)(-13.5)}}{20} = \frac{-2 + \sqrt{4 + 540}}{20} = \frac{-2 + \sqrt{544}}{20} \approx \frac{-2 + 23.32}{20} = 1.066$$

Suy ra $I_D = x^2 = (1.066)^2 \approx \mathbf{1.136 \text{ mA}}$.

\textbf{Kiểm tra điều kiện $V_{GS} > V_t$:}

$V_D = 15 - 10(1.136) = 15 - 11.36 = 3.64$ V.

$V_{GS} = V_D = 3.64 \text{ V} > 1.5 \text{ V}$ (Thỏa mãn).

\textbf{Kết quả câu a:}
\begin{itemize}
    \item Dòng máng: $I_D \approx 1.136$ mA
    \item Dòng cổng: $I_G = 0$ A
    \item Dòng nguồn: $I_S = I_D \approx 1.136$ mA
\end{itemize}

\subsection*{b) Tính thông số $g_m$ và $r_o$ của FET}

\textbf{Độ dẫn truyền $g_m$:}

Sử dụng công thức $g_m = 2k_n(V_{GS} - V_t)$ với các giá trị DC đã tính.

$$g_m = 2 \times (0.25 \text{ mA/V}^2) \times (3.64 \text{ V} - 1.5 \text{ V}) = 0.5 \times 2.14 = \mathbf{1.07 \text{ mA/V}}$$

(hoặc mS).

\textbf{Điện trở ngõ ra $r_o$:}

Sử dụng công thức $r_o = \frac{V_A}{I_D}$.

$$r_o = \frac{57 \text{ V}}{1.136 \text{ mA}} \approx \mathbf{50.18 \text{ k}\Omega}$$

\subsection*{c) Tính tỉ số độ lợi áp $A_v = v_o / v_i$}

\textbf{Sơ đồ tương đương tín hiệu nhỏ (AC):}

Nguồn DC $V_{DD}$ nối đất. Tụ điện ngắn mạch.

Đây là mạch Source chung với điện trở hồi tiếp $R_G$ từ Drain về Gate.

Tại ngõ ra (cực D), ta có ba điện trở mắc song song: $R_D$, $R_L$, và $r_o$ của FET.

Gọi $R_L' = R_D \parallel R_L \parallel r_o$.

$$\frac{1}{R_L'} = \frac{1}{10} + \frac{1}{10} + \frac{1}{50.18} \approx 0.1 + 0.1 + 0.0199 = 0.2199 \text{ (k}\Omega)^{-1}$$

$$R_L' = \frac{1}{0.2199} \approx 4.55 \text{ k}\Omega$$

\textbf{Tính độ lợi áp:}

Áp dụng định luật KCL tại nút Drain (ngõ ra $v_o$):

Dòng từ $R_G$ + Dòng từ nguồn dòng $g_m$ + Dòng qua $R_L'$ = 0

$$\frac{v_i - v_o}{R_G} - g_m v_{gs} - \frac{v_o}{R_L'} = 0$$

Với $v_{gs} = v_i$ (do cực Source nối đất):

$$\frac{v_i}{R_G} - \frac{v_o}{R_G} - g_m v_i - \frac{v_o}{R_L'} = 0$$

$$v_i (\frac{1}{R_G} - g_m) = v_o (\frac{1}{R_G} + \frac{1}{R_L'})$$

Tỉ số độ lợi áp:

$$A_v = \frac{v_o}{v_i} = \frac{\frac{1}{R_G} - g_m}{\frac{1}{R_G} + \frac{1}{R_L'}}$$

Vì $R_G = 10$ M$\Omega$ rất lớn, nên $\frac{1}{R_G} \approx 0$. Công thức gần đúng là:

$$A_v \approx \frac{-g_m}{1/R_L'} = -g_m R_L'$$

Thay số:

$$A_v \approx -(1.07 \text{ mA/V}) \times (4.55 \text{ k}\Omega) \approx \mathbf{-4.87}$$

Kết quả câu c: $A_v \approx -4.87$. Dấu âm cho thấy tín hiệu ngõ ra đảo pha so với ngõ vào.

\subsection*{d) Tính trở kháng ngõ vào $R_{in}$}

Do có điện trở $R_G$ nối giữa ngõ vào (Gate) và ngõ ra (Drain), ta áp dụng định lý Miller.

\textbf{Công thức Miller cho trở kháng ngõ vào:}

Điện trở $R_G$ nhìn từ ngõ vào sẽ có giá trị tương đương là:

$$R_{in} = \frac{R_G}{1 - A_v}$$

\textbf{Tính toán:}

Thay các giá trị $R_G = 10$ M$\Omega$ và $A_v = -4.87$ vào công thức:

$$R_{in} = \frac{10 \text{ M}\Omega}{1 - (-4.87)} = \frac{10 \text{ M}\Omega}{1 + 4.87} = \frac{10}{5.87} \text{ M}\Omega \approx \mathbf{1.70 \text{ M}\Omega}$$

Kết quả câu d: $R_{in} \approx 1.70$ M$\Omega$.

