\documentclass[12pt,a4paper]{article}

% ==================== PACKAGES ====================
\usepackage{fontspec}
\usepackage[english,vietnamese]{babel}
\setmainfont{Latin Modern Roman}
\usepackage{amsmath,amssymb,amsfonts}
\usepackage{graphicx}
\usepackage{geometry}
\usepackage{hyperref}
\usepackage{booktabs}
\usepackage{array}

% ==================== PAGE SETUP ====================
\geometry{
    left=3cm,
    right=2cm,
    top=2.5cm,
    bottom=2.5cm
}

% ==================== HYPERREF SETUP ====================
\hypersetup{
    colorlinks=true,
    linkcolor=blue,
    filecolor=magenta,
    urlcolor=cyan,
    citecolor=red
}

% ==================== DOCUMENT ====================
\begin{document}

\selectlanguage{vietnamese}

\title{Câu 1}
\author{}
\date{}
\maketitle

\section*{Câu 1}

\subsection*{a) Tính giá trị các dòng của BJT ở chế độ DC}

Ở chế độ DC, các tụ điện $C_{C1}, C_{C2}$ hở mạch. Nhìn vào sơ đồ, chân Base (B) được nối đất ($V_B = 0$ V). Đây là mạch khuếch đại kiểu Base chung (Common Base) dùng transistor PNP.

\textbf{Tính dòng phát (Emitter current - $I_E$):}

Áp dụng định luật Kirchoff cho vòng từ nguồn $V^+$ xuống Ground qua mối nối E-B:

$$V^+ - I_E R_E - V_{EB} = V_B$$

Thay số ($V_B = 0, V_{EB} = V_{\gamma} = 0.67$ V):

$$10 - I_E(10\text{k}) - 0.67 = 0 \Rightarrow I_E = \frac{10 - 0.67}{10\text{k}} = \mathbf{0.933 \text{ mA}}$$

\textbf{Tính dòng thu (Collector current - $I_C$):}

$$I_C = \alpha I_E = \left( \frac{\beta}{\beta + 1} \right) I_E = \left( \frac{100}{101} \right) \times 0.933 \approx \mathbf{0.924 \text{ mA}}$$

\textbf{Tính dòng cực nền (Base current - $I_B$):}

$$I_B = \frac{I_C}{\beta} = \frac{0.924}{100} = 0.00924 \text{ mA} = \mathbf{9.24 \text{ }\mu\text{A}}$$

\subsection*{b) Vẽ sơ đồ tương đương tín hiệu nhỏ}

\begin{figure}[h]
\centering
\includegraphics[width=0.8\textwidth]{images/small_signal_cb.png}
\caption{Sơ đồ tương đương tín hiệu nhỏ mạch khuếch đại BJT Base chung (Common Base)}
\label{fig:small_signal}
\end{figure}

Ở chế độ AC (tín hiệu nhỏ):
\begin{itemize}
    \item Các tụ điện $C_{C1}, C_{C2}$ coi như ngắn mạch.
    \item Các nguồn DC ($V^+, V^-$) được nối đất.
    \item Sử dụng mô hình T-model (rất phù hợp cho mạch Base chung).
\end{itemize}

\textbf{Mô tả sơ đồ:}
\begin{itemize}
    \item Ngõ vào ($v_i$): Nối vào cực E. Điện trở $R_E$ lúc này mắc song song với ngõ vào (nối từ E xuống đất).
    \item Cực Base (B): Nối trực tiếp xuống đất (do $C_{C2}$ ngắn mạch).
    \item Cực Collector (C): Có nguồn dòng phụ thuộc $\alpha i_e$ hướng từ C sang B (với PNP) và điện trở $R_C$ nối xuống đất.
    \item Ngõ ra ($v_o$): Lấy tại cực C.
\end{itemize}

\subsection*{c) Tính tỉ số độ lợi áp $A_v = v_o / v_i$ (Không tải)}

Trước hết, tính điện trở phát động $r_e$:

$$r_e = \frac{V_T}{I_E} = \frac{26 \text{ mV}}{0.933 \text{ mA}} \approx 27.87 \Omega$$

Trong mạch Base chung, điện áp ngõ vào đặt vào cực E: $v_i = v_e$.

Dòng tín hiệu nhỏ tại cực phát là $i_e = \frac{v_i}{r_e}$ (chạy từ E vào).

Dòng tại cực thu là $i_c = \alpha i_e$.

Điện áp ngõ ra tại cực C (không tải $R_L$):

$$v_o = i_c \cdot R_C = (\alpha \frac{v_i}{r_e}) \cdot R_C$$

Độ lợi áp:

$$A_v = \frac{v_o}{v_i} = \frac{\alpha \cdot R_C}{r_e} = \frac{0.99 \times 5000}{27.87} \approx \mathbf{177.6}$$

(Lưu ý: Mạch Base chung không đảo pha nên $A_v$ mang dấu dương).

\subsection*{d) Tính lại $A_v$ khi gắn thêm tải $R_L = 17$ k$\Omega$}

Khi gắn thêm tải $R_L$ qua tụ $C_{C2}$ (lúc này $C_{C2}$ ngắn mạch AC), điện trở tương đương tại cực thu sẽ là $R_C$ mắc song song với $R_L$.

\textbf{Tính điện trở tải tương đương ($R_C'$):}

$$R_C' = R_C \parallel R_L = \frac{5 \times 17}{5 + 17} = \frac{85}{22} \approx 3.864 \text{ k}\Omega$$

\textbf{Tính độ lợi áp mới:}

$$A_v = \frac{\alpha \cdot R_C'}{r_e} = \frac{0.99 \times 3864}{27.87} \approx \mathbf{137.3}$$

\subsection*{Kết quả tổng hợp:}

\begin{table}[h]
\centering
\begin{tabular}{@{}ll@{}}
\toprule
\textbf{Thông số} & \textbf{Giá trị} \\
\midrule
$I_C$ (DC) & $0.924$ mA \\
$A_v$ (Không tải) & $177.6$ \\
$A_v$ (Có tải $17$k) & $137.3$ \\
\bottomrule
\end{tabular}
\end{table}

\end{document}
